\documentclass[11pt,a4paper]{article}
%\usepackage[dutch]{babel} % This is for Dutch layout rules. 

\usepackage[round]{natbib} % formats references

\usepackage{graphics} % for graphs in eps files

\usepackage{amsmath}         % Additional math symbols


\begin{document}
\title{Group 007\\Our super-descriptive title for the project}
\author{An Author\\ERNA number
\and Another Author\\ERNA number
\and A third Author\\ERNA number
\and And me\\ERNA number}
\date{Final Report: General Econometrics}

\maketitle

\begin{abstract}
The abstract helps the reader to determine the relevance of the article. 
It contains a short paragraph that describes the goals, methods, results, 
and conclusion.
\end{abstract}


\section{Introduction}
The topic of our research is of relevance because \ldots and this is what we want to find out specifically. It ties in with the existing literature, such as \cite{amemiya85}, who does a lot in an outstanding book, and \cite{carter97}, who do other interesting things. In contrast to the literature, we do it all differently and better.

\section{Methods}
Now this is what we are going to do. We use equations as follows
\begin{equation}
    y_i =\mathbf{x}_i^\prime\boldsymbol\beta + \varepsilon_i
\label{eq:our_first_model}.
\end{equation}
where the symbol have the following meaning \ldots

And we can make a reference to our first models as~\eqref{eq:our_first_model}. In case we put another equation before this one the cross-reference will still work. How cool are computers!


\section{Results}
Present your results here and then discuss them. All that is relevant is discussed. Nothing is left undiscovered. Important results are can be seen in Table~\ref{tab:a_great_table}. 

You can also tie in figures. If you your plain \LaTeX\ you will have to use eps files (easy with Matlab, which you are certainly using). 
PDF\LaTeX\ can also work with other formats such as jpg, png, and pdf.



\begin{table}[t]
\begin{center}
\caption{A descriptive title}
\label{tab:a_great_table}
\begin{tabular}{lc}
\hline\hline
Col 1 & Col 2\\
\hline
Entry 1 & Entry 2\\
\hline\hline
\multicolumn{2}{l}{\footnotesize{Note: Notes here}}
\end{tabular}
\end{center}
\end{table}

\section{Conclusion}
Draw conclusions and connect those to the existing literature.



\bibliography{Basisweco}
\bibliographystyle{apalike}
\end{document}
